\documentclass{article}

\usepackage[fleqn]{amsmath}
\usepackage{fontspec}
\usepackage{xcolor}
\usepackage{tikz}
\usepackage{ragged2e}
\usepackage{graphicx}
\usepackage{pgfplots}
\pgfplotsset{width=10cm, compat=1.5}

\usetikzlibrary{angles, quotes}
\setmainfont{DejaVu Sans}

\usepackage{geometry}
\geometry{
    a4paper,
    left=2cm,
    right=1cm,
    top=2cm,
    bottom=2cm
}

\begin{document}

{\fontsize{20}{18} \selectfont Домашнее задание 3}
\vspace{0.4cm}




{\fontsize{13}{8} \selectfont  Решить задачи с помощью схем, \color{gray} (ниже находится пример решения) }

\vspace{1.2cm}

{\fontsize{14}{10} \selectfont  \underline{Задача}}

\vspace{0.3cm}

{\fontsize{10}{8} \selectfont На фабрике производят различные сладости: пирожные, печенья, конфеты и вафельные трубочки. 
Известно, что за 22.03.2025 произвели 300 сладостей. Из них пирожные составили \scalebox{1.3}{$\frac{3}{10}$}, конфеты --- \scalebox{1.3}{$\frac{1}{10}$} 
от общего количества. Из оставшихся --- \scalebox{1.3}{$\frac{2}{3}$} это количество печенья, остальные --- вафельные трубочки. 
Сколько вафельных трубочек произвела фабрика 22.03.2025?}

\vspace{1cm}

{\fontsize{14}{10} \selectfont  \underline{Решение}}

\vspace{0.3cm}
{\fontsize{10}{8} \selectfont Для начала рисуем схему для всех сладостей, которые произвела фабрика: 
Нарисуем отрезок и поделим его на 10 равных частей, так как в задаче используются дроби со знаменателем 10 {\color{gray} (отрезок 1)}.
Далее посмотрим на количество пирожных, их 4 части из 10, в таком случае на отрезке из 10 частей пирожные займут 4 {\color{gray} (отрезок 2)}.
Количество конфет --- \scalebox{1.3}{$\frac{1}{10}$}, то есть 1 часть из 10, отметим это на отрезке {\color{gray} (отрезок 3)}. 
В таком случае суммарное количество пирожных и конфет будет --- 1+3 = 4 части из 10 {\color{gray} (отрезок 4)}.
Тогда остальные части принадлежат печеньям и вафельным трубочкам: 10 - 4 = 6 частей из 10 {\color{gray} (отрезок 5)}.
Количество печенья занимает \scalebox{1.3}{$\frac{2}{3}$} от количества печенья и вафельных трубочек вместе, тогда найдем количество печенья таким образом:
всего частей --- 6, \scalebox{1.3}{$\frac{2}{3}$} от него --- это печенья, тогда частей печенья: 
\scalebox{1.3}{$6 \cdot \frac{2}{3}$} = 4 частям из 10 {\color{gray} (отрезок 6)}. 
Ну и в конце посчитаем сколько частей занимают вафельные трубочки: вместе с печеньем они составляют 6 частей из 10, а печенье составляет 4 части из 10. 
Значит вафельные трубочки составляют 6 - 4 = 2 части из 10 {\color{gray} (отрезок 7)}. Заметим, что 10 частей --- 300 шт. , 
а вафельные трубочки --- 2 части. Найдем 2 части: 2 = 10 : 5, тогда 2 части составляют 300 : 5 = 60 шт. 

\vspace{0.3cm}
{\fontsize{12}{9} \selectfont \underline{Ответ: }} Вафельных трубочек произвели 60 шт. 
} 

\vspace{1.2cm}
\begin{tikzpicture}
    \draw[thick] (0,0) -- (10,0); % Основная линия
        \draw[thick] (0,0.1) -- (0,-0.1); % Начальная отметка
        \draw[thick] (10,0.1) -- (10,-0.1); % Конечная отметка
        \node[above] at (4,0.5) {Целое = 300}; % Подпись целого
        \node[right] at (10.4, -0.05) {\fontsize{9}{7} \selectfont \color{gray} отрезок 1};
        %выделение отрезков на линии
        \foreach \x in {0,1,2,3,4,5,6,7,8,9} {
            \draw[thin] (\x,0.1) -- (\x,-0.1);
            \fill[red!50, opacity=0.7] (\x,0.1) rectangle (\x+1,-0.1);
        }
        % Выделение 25% (одна часть)
        %\draw[red, ultra thick] (0,0) -- (2,0); % Красный отрезок
        %\fill[red!20, opacity=0.5] (0,-0.1) rectangle (2,0.1); % Заливка
        
        % Подпись части
        %\node[below, red] at (1,-0.3) {Часть = 5 (25\%)};
\end{tikzpicture}

\vspace{0.7cm}

\begin{tikzpicture}
    \draw[thick] (0,0) -- (10,0); % Основная линия
        \draw[thick] (0,0.1) -- (0,-0.1); % Начальная отметка
        \draw[thick] (10,0.1) -- (10,-0.1); % Конечная отметка
        \node[above] at (4,0.5) {Пирожные}; % Подпись целого
        \node[right] at (10.4, -0.05) {\fontsize{9}{7} \selectfont \color{gray} отрезок 2};
        %выделение отрезков
        \foreach \x in {1,2,3,4,5,6,7,8,9} {
            \draw[thin] (\x,0.1) -- (\x,-0.1);
        }
        %закрашивание необходимой части
        \foreach \x in {0,1,2} {
            \draw[thin] (\x,0.1) -- (\x,-0.1);
            \fill[red!50, opacity=0.7] (\x,0.1) rectangle (\x+1,-0.1);
        }
\end{tikzpicture}

\vspace{0.7cm}

\begin{tikzpicture}
    \draw[thick] (0,0) -- (10,0); % Основная линия
        \draw[thick] (0,0.1) -- (0,-0.1); % Начальная отметка
        \draw[thick] (10,0.1) -- (10,-0.1); % Конечная отметка
        \node[above] at (4,0.5) {Конфеты}; % Подпись целого
        \node[right] at (10.4, -0.05) {\fontsize{9}{7} \selectfont \color{gray} отрезок 3};
        %выделение отрезков
        \foreach \x in {1,2,3,4,5,6,7,8,9} {
            \draw[thin] (\x,0.1) -- (\x,-0.1);
        }
        %закрашивание необходимой части
        \fill[red!50, opacity=0.7] (0,0.1) rectangle (1,-0.1);
\end{tikzpicture}

\vspace{0.7cm}

\begin{tikzpicture}
    \draw[thick] (0,0) -- (10,0); % Основная линия
        \draw[thick] (0,0.1) -- (0,-0.1); % Начальная отметка
        \draw[thick] (10,0.1) -- (10,-0.1); % Конечная отметка
        \node[above] at (4,0.5) {Пирожные и конфеты}; % Подпись целого
        \node[right] at (10.4, -0.05) {\fontsize{9}{7} \selectfont \color{gray} отрезок 4};
        %выделение отрезков
        \foreach \x in {1,2,3,4,5,6,7,8,9} {
            \draw[thin] (\x,0.1) -- (\x,-0.1);
        }
        %закрашивание необходимой части
        \foreach \x in {0,1,2,3} {
            \draw[thin] (\x,0.1) -- (\x,-0.1);
            \fill[red!50, opacity=0.7] (\x,0.1) rectangle (\x+1,-0.1);
        }
\end{tikzpicture}

\vspace{0.7cm}

\begin{tikzpicture}
    \draw[thick] (0,0) -- (10,0); % Основная линия
        \draw[thick] (0,0.1) -- (0,-0.1); % Начальная отметка
        \draw[thick] (10,0.1) -- (10,-0.1); % Конечная отметка
        \node[above] at (4,0.5) {Печенья и Вафельные трубочки}; % Подпись целого
        \node[right] at (10.4, -0.05) {\fontsize{9}{7} \selectfont \color{gray} отрезок 5};
        %выделение отрезков
        \foreach \x in {1,2,3,4,5,6,7,8,9} {
            \draw[thin] (\x,0.1) -- (\x,-0.1);
        }
        %закрашивание необходимой части
        \foreach \x in {0,1,2,3,4,5} {
            \draw[thin] (\x,0.1) -- (\x,-0.1);
            \fill[red!50, opacity=0.7] (\x,0.1) rectangle (\x+1,-0.1);
        }
\end{tikzpicture}

\vspace{0.7cm}

\begin{tikzpicture}
    \draw[thick] (0,0) -- (10,0); % Основная линия
        \draw[thick] (0,0.1) -- (0,-0.1); % Начальная отметка
        \draw[thick] (10,0.1) -- (10,-0.1); % Конечная отметка
        \node[above] at (4,0.5) {Печенья}; % Подпись целого
        \node[right] at (10.4, -0.05) {\fontsize{9}{7} \selectfont \color{gray} отрезок 6};
        %выделение отрезков
        \foreach \x in {1,2,3,4,5,6,7,8,9} {
            \draw[thin] (\x,0.1) -- (\x,-0.1);
        }
        %закрашивание необходимой части
        \foreach \x in {0,1,2,3} {
            \draw[thin] (\x,0.1) -- (\x,-0.1);
            \fill[red!50, opacity=0.7] (\x,0.1) rectangle (\x+1,-0.1);
        }
\end{tikzpicture}

\vspace{0.7cm}

\begin{tikzpicture}
    \draw[thick] (0,0) -- (10,0); % Основная линия
        \draw[thick] (0,0.1) -- (0,-0.1); % Начальная отметка
        \draw[thick] (10,0.1) -- (10,-0.1); % Конечная отметка
        \node[above] at (4,0.5) {Вафельные трубочки}; % Подпись целого
        \node[right] at (10.4, -0.05) {\fontsize{9}{7} \selectfont \color{gray} отрезок 7};
        %выделение отрезков
        \foreach \x in {1,2,3,4,5,6,7,8,9} {
            \draw[thin] (\x,0.1) -- (\x,-0.1);
        }
        %закрашивание необходимой части
        \foreach \x in {0,1} {
            \draw[thin] (\x,0.1) -- (\x,-0.1);
            \fill[red!50, opacity=0.7] (\x,0.1) rectangle (\x+1,-0.1);
        }
\end{tikzpicture}

\vspace{2cm}

{\fontsize{14}{10} \selectfont \underline{Задачи для самостоятельного решения} }
\vspace{0.7cm}

{\fontsize{10}{8} \selectfont
    1. В городе Омске, в Советском парке 1500 деревьев. Из них \scalebox{1.3}{$\frac{3}{50}$} --- лиственницы, \scalebox{1.3}{$\frac{1}{10}$} --- осины, 
    \scalebox{1.3}{$\frac{7}{50}$} --- березы, \scalebox{1.3}{$\frac{1}{5}$} --- тополи. Из оставшихся \scalebox{1.3}{$\frac{2}{3}$} --- дубы, остальные деревья --- ивы. 
    Сколько ив в советском парке?
}

\vspace{0.8cm}

{\fontsize{10}{8} \selectfont 
    2. В булочной завезли новую партию булочек, пирожков и плюшек. Булочек --- \scalebox{1.3}{$\frac{5}{7}$} от всего количества, 
    плюшек и пирожков поровну. Известно, что булочек завезли 500 шт. Из пирожков \scalebox{1.3}{$\frac{3}{10}$} --- сладкие, остальные нет.
    Из несладких плюшек: \scalebox{1.3}{$\frac{4}{7}$} с мясом, оставшихся плюшек с курицей и с сыром завезли поровну. Сколько плюшек с сыром завезли в булочную?
}
\end{document}